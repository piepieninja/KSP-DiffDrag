
\documentclass[journal,a4paper]{IEEEtran}

\usepackage{graphicx}
\usepackage{tikz}
\usepackage[margin=1.5cm]{geometry}
\usepackage[most]{tcolorbox}
\usepackage{chessboard}
\usepackage{float}
\usepackage{hyperref}
\usepackage{url}
%\usepackage{biblatex}
\usepackage{cite}
\usepackage{setspace}

%
% If IEEEtran.cls has not been installed into the LaTeX system files,
% manually specify the path to it like:
% \documentclass[journal]{../sty/IEEEtran}

% correct bad hyphenation here
\hyphenation{op-tical net-works semi-conduc-tor}



\doublespacing
\begin{document}
%
% paper title
% can use linebreaks \\ within to get better formatting as desired
\title{Towards Differential Drag Control for\\ Autonomous Multiagent Satellites in\\ Kerbal Space Program}

\author{
\textbf{Caleb Ashmore Adams}\\
\textit{Department of Computer Science}, University of Georgia\\
\textit{Small Satellite Research Laboratory}, University of Georgia\\
1510 Cedar St.\\
Athens, GA 30602\\
770-314-8422\\
pieninja@uga.edu
}


% The paper headers
\markboth{UGA CSCI/ARTI 6550 Term Project}%
{Shell \MakeLowercase{\textit{et al.}}: The Title of your Paper}
% The only time the second header will appear is for the odd numbered pages
% after the title page when using the twoside option.
%

% make the title area
\maketitle

% The Abstract
\begin{abstract}
  The concept of differential drag has received a significant amount of research
  attention due to its relevance with emerging space technologies and its potential
  use in satellite swarms \cite{horsley}. Furthermore, the computational complexity
  of many differential drag based satellite swarms makes them ideal for automation
  through Artificial Intelligence (AI) \cite{swarm_ai}. However, this computational
  complexity - coupled with the necessity for high fidelity physics simulations -
  quickly becomes a long and arduous challenge filled with various optimization
  problems \cite{sin}. Due to the intended brevity of this research, simpler models
  are preferable. Humorously, this makes the popular aerospace engineering simulation
  game, Kerbal Space Program (KSP), an ideal candidate due to its simplicity and
  customizability (mods). AI does, after all, have a wonderful history of being
  applied to video games for the aforementioned reasons \cite{ai_games_book}\cite{wiki_ai_games}.
  Thus, this project seeks to implement autonomous formation flying and rendezvous
  in KSP using mods, Python, and C\#. This research will roughly follow
  Russell and Norvig’s Planning and Moving (chapter  25.4, 25.5, and 25.6) and Multiagent
  Planning (chapter 11.4) \cite{class_book}.

\end{abstract}

\IEEEpeerreviewmaketitle


% ==============================================================================
%
% section
%
% ==============================================================================
\section{Introduction}

% =====================================
% subsection
% =====================================
\subsection{Kerbal Space Program}
asdf

% =====================================
% subsection
% =====================================
\subsection{Modding KSP}
asdf

% =====================================
% subsection
% =====================================
\subsection{Differential Drag}
Put simply, drag is a force that acts opposite to the relative motion of any object
moving with respect to a surrounding fluid \cite{drag_deff}. For the purposes of this
research I will attempt to simplify our interpretation of drag as much as possible.
The drag equation is as follows:
\begin{equation}
F_D = \frac{1}{2} \rho u_a^2 C_D A
\end{equation}
In formula $1$, the drag equation, $F_D$ is the force of drag acting on the object.
$\rho$ is the mass density of the fluid the object moves through and $u$ is the relative
velocity of the fluid immediatley sourrounding the object. $C_D$ is the coeffecient of
drag of the object and $A$ is the surface area of the object that collides with
partiles in the fluid. Low Earth orbit, in our case low Kerbin orbit, has a very low $\rho$ value. Interestingly,
the density of the atmosphere at $400km$ is still significant enough to
generate non-negligible drag on spacecraft. The term Differential Drag
comes from the fact that a spacecraft can reorient itself to increase the surface
Area, $A$, it has exposes to particles within the atmosphere. This means that any
spacecraft capible of reorienting itself is also capible of generating a maximum
and a minimum amount drag
The primary motivation for this research was to see if KSP had similar physcis and
if I could design a Differential Drag system within the game. To preform Differential
Drag in KSP it in nessesary to calculate
these values in real time to generate the force $F_D$. It is also nessesary to




% ==============================================================================
%
% section
%
% ==============================================================================
\section{Multiagent Satellites}
asdf

% =====================================
% subsection
% =====================================
\subsection{The Rendezvous Problem}
Lin et al state that the multiagent rendezvous problem is to devise "local" control
strategies, one for each agent, which without any active communication between agents,
cause all members of the group to eventually rendezvous at single unspecified location \cite{lin_multi}.
Furthermore, Russell and Norvig discribe three particular values that any such multiagent
system is likely to have per agent \cite{class_book}:

\begin{enumerate}
  \item Cohesion, a positive score for getting closer to the average position of the neighbors
  \item Separation, a negative score for getting too close to any one neighbor
  \item Alignment, a positive score for getting closer to the average heading of the neighbors
\end{enumerate}

With the given descriptions one can easily imagine a purely cohesion-based rendezvous
system where agents converge to some middle point. If we view a given agent as a
particle $p$ placed randomly within a place, we can use use each particle's $(x,y)$
value to calculate a centeroid. The unit vector, at each particle, pointing to the
centeroid represents the movement for the particle that results in the most immediate
cohesion.

\begin{equation}
  C(x,y) = (\frac{x_1 + x_2 \hspace{1mm} ... + x_n}{n} , \frac{y_1 + y_2 ... \hspace{1mm} + y_n}{n})
\end{equation}

I have provided an example program under \texttt{src/examples/scatter\_join.py} that
performs the above functions. However, this alone is not sufficent. Lin further argues
that robust multiagent rendezvous requires a logical limit on the sensing capabilities
of an individual agent \cite{lin_multi}.


% \subsubsection{Current Solutions}
% asdf
%
% \subsubsection{My Example Solution}
% asdf

% =====================================
% subsection
% =====================================
\subsection{The Real Rendezvous Problem}
adsf

\subsubsection{Current Solutions}
asdf

\subsubsection{My Example Solution}
adsf

% ==============================================================================
%
% section
%
% ==============================================================================
\section{Analysis}
asdf

% =====================================
% subsection
% =====================================
\subsection{Functional Rendezvous}
asdf

% =====================================
% subsection
% =====================================
\subsection{Issues with KSP Drag}
asdf

% =====================================
% subsection
% =====================================
\subsection{Conclusion}
asdf

% =====================================
% subsection
% =====================================
\subsection{Future Work}


% \appendices
% \section{Summarizing Maxwell Equations}
% Appendix one text goes here.
%
% % you can choose not to have a title for an appendix
% % if you want by leaving the argument blank
% \section{}
% Appendix two text goes here.
%
%
% % use section* for acknowledgement
% \section*{Acknowledgment}

\bibliographystyle{IEEEtran}
\bibliography{IEEEabrv,report}{}


% \begin{thebibliography}{1}

% \bibitem{IEEEhowto:kopka}
% H.~Kopka and P.~W. Daly, \emph{A Guide to \LaTeX}, 3rd~ed.\hskip 1em plus
%   0.5em minus 0.4em\relax Harlow, England: Addison-Wesley, 1999.
%
% \end{thebibliography}



% that's all folks
\end{document}
